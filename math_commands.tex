%% Commande supplémentaire

% semi-norm of a vector
\newcommand{\snorm}[1]{\left| #1 \right|}
% norm of a vector
\newcommand{\norm}[1]{\left\| #1 \right\|}
% degree symbol
\newcommand{\degree}[0]{^\circ} 
% rename builtin command \v{} to \vaccent{}
\let\vaccent=\v 
% for vectors
\renewcommand{\v}[1]{\ensuremath{\mathbf{#1}}} 
% for vectors of Greek letters
\newcommand{\gv}[1]{\ensuremath{\mbox{\boldmath$ #1 $}}} 
% for unit vector
\newcommand{\uv}[1]{\ensuremath{\mathbf{\hat{#1}}}}	
% for tensors
%\newcommand{\tn}[1]{\ensuremath{\pmb{\mathsf{#1}}}} 
% for tensors of Greek letters
\newcommand{\tn}[1]{\ensuremath{\mbox{\boldmath$\mathsf{#1}$}}} 
% for absolute value
\newcommand{\abs}[1]{\left| #1 \right|}	
% for average
\newcommand{\avg}[1]{\left< #1 \right>} 
% rename builtin command \d{} to \underdot{}
\let\underdot=\d 
% for derivatives
\renewcommand{\d}[2]{\frac{d #1}{d #2}} 
% for double derivatives
\newcommand{\dd}[2]{\frac{d^2 #1}{d #2^2}} 
% for partial derivatives
\newcommand{\pd}[2]{\frac{\partial #1}{\partial #2}}
% for double partial derivatives
\newcommand{\pdd}[2]{\frac{\partial^2 #1}{\partial #2^2}} 
% for crossed double partial derivatives
\newcommand{\pdx}[3]{\frac{\partial^2 #1}{\partial #2 \partial #3}} 
% for thermodynamic partial derivatives
\newcommand{\pdc}[3]{\left( \frac{\partial #1}{\partial #2} \right)_{#3}} 
% for Dirac bras
\newcommand{\ket}[1]{\left| #1 \right>} 
% for Dirac kets
\newcommand{\bra}[1]{\left< #1 \right|} 
% for Dirac brackets
\newcommand{\braket}[2]{\left< #1 \vphantom{#2} \right| \left. #2 \vphantom{#1} \right>} 
% for Dirac matrix elements
\newcommand{\matrixel}[3]{\left< #1 \vphantom{#2#3} \right| #2 \left| #3 \vphantom{#1#2} \right>} 
% for gradient
\newcommand{\grad}[1]{\gv{\nabla} #1}
% rename builtin command \div to \divsymb
\let\divsymb=\div 
% for divergence
\renewcommand{\div}[1]{\gv{\nabla} \cdot #1} 
% for curl
\newcommand{\curl}[1]{\gv{\nabla} \times #1} 
% for laplacian
\newcommand{\lap}[1]{\gv{\nabla}^2 #1}
% Math text
\newcommand{\mt}[1]{\mathrm{#1}} 
% Overline
\newcommand{\ol}[1]{\overline{#1}}
% Partial derivative with dfrac
\newcommand{\dpd}[2]{\dfrac{\partial #1}{\partial #2}}
% Scientific notation
\providecommand{\e}[1]{\ensuremath{\times 10^{#1}}}
% Symbole degré
\renewcommand{\deg}{$^{\circ}\;$}

% text style
%
\newcommand{\Ae}{\text{\normalshape a.e.}}
% c'est-à-dire
\newcommand{\ie}{\textit{i.e.}\ }
% e.g. - par exemple
\newcommand{\eg}{\textit{e.g.}\ }
\newcommand{\strong}{\text{\normalshape -strong}}
\newcommand{\weak}{\text{\normalshape -weak}}
\newcommand{\loc}{\text{\normalshape loc}}
\newcommand{\ad}{\text{\normalshape ad}}
% Symbole degré
\renewcommand{\deg}{$^{\circ}\;$}
% Accronymes
\newcommand{\PIV}{\textit{PIV}\ }

% Space operator
\newcommand{\Def}{\overset{\text{\textup{def}}}{=}}
\newcommand{\R}{\operatorname{\mathbb R}}
\newcommand{\Rn}{\operatorname{{\mathbb R}^N}}
\newcommand{\RK}{\operatorname{{\mathbb R}^K}}
\newcommand{\N}{\operatorname{\mathbb N}}
\newcommand{\Z}{\operatorname{\mathbb Z}}
\newcommand{\ON}{\operatorname{\text{\textup{O(N)}}}}
\newcommand{\opt}{\mathrm{opt}}

% Matrix operator
\newcommand{\transp}{\:{}^*\,\negmedspace}
\newcommand{\trans}{\:{}^* \negmedspace}
\newcommand{\transm}{\:{}^* \!\negmedspace}
\newcommand{\tran}{{}^* \negmedspace}

% Environnement pour les Théorème, lemmes et remarques
\newtheorem{theoreme}{Th\'{e}or\`{e}me}
\newtheorem{lemme}{Lemme}
\newtheorem{remarque}{Remarque}
\newtheorem{prop}{Proposition}
\newtheorem{hypo}{Hypoth\`{e}se}
\newtheorem{dfn}{D\'{e}finition}

% Variables
\newcommand{\constant}[1]{\mathit{#1}}
\renewcommand\Re{\constant{Re}}  % Reynolds number
\newcommand\Pe{\constant{Pe}}  % Peclet number
\newcommand\Nu{\constant{Nu}}  % Nusselt number
\newcommand\Sh{\constant{Sh}}  % Sherwood number
\newcommand\Sc{\constant{Sc}}  % Schmidt number
\renewcommand\Pr{\constant{Pr}}  % Prandlt number
% Trucs de chimie
\newcommand\el{\mathrm{e^-}}
\newcommand*\chem[1]{\ensuremath{\mathrm{#1}}}